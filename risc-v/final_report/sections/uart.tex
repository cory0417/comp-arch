\section{UART}

The UART peripheral was implemented with simply the receiver, which was connected to a FIFO. The transaction configurations were assumed to be 115200 baud, 8 data bits, no parity, and 1 stop bit, which is a common configuration. With the existing system clock of 12 MHz, the receiver samples the incoming signal at every 104 clock cycles (12 MHz / 115200 baud \tilde=104.17). When the receiver detects a start bit (active low), it starts to sample the incoming signal and first offsets the sampling point by 52 clock cycles (half the baud period) to ensure that the middle of the start bit is sampled. Then, the receiver samples the incoming signal at every baud period and uses a majority voting logic to determine the value of the signal. When the stop bit is asserted, the receiver asserts a data ready signal for transfer to the FIFO. The UART controller then acknowledges the data ready signal and the FIFO is updated. 

When the FIFO is full, the UART controller asserts a FIFO full signal, which is connected to the reset signal of the core. This effectively pauses the core and waits for the top-level module to clear the FIFO. The program is transferred to the instruction memory and the core is released from reset. 
